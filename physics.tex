


 The current highest precision measurement for the mass and width are results of measurements from both WW production at $e^+ e^-$ colliders and single $W^\pm$ production at hadron colliders. These measurements use the combined results from LEP, Tevatron, and LHC experiments which report $M_W = 80.379 \pm 0.012 \, \, \text{GeV} $ and $\Gamma_W = 2.085 \pm 0.042 \,  \,\text{GeV}$ \cite{pdg}. The diagrams representing WW production are given in Figure \ref{fig:wwdiag}. The final states of the WW process are either the fully hadronic $WW\rightarrow q\bar{q}q\bar{q}$ which comprises $46\%$ of the total WW cross-section, semileptonic $WW\rightarrow q\bar{q}\ell\nu_{\ell}$ which comprises $44\%$, or fully leptonic $WW\rightarrow \ell \nu \ell \nu$ filling the remaining cross-section \cite{wwOPAL}. The semileptonic mode is the most favorable way to measure the W mass because the hadronic system is easily obtained after the identification of the lepton. The semileptonic channel also offers a large contribution to the $q\bar{q}e\nu_e$ final state through hadronic single W's alongside a non-resonant $e \nu_e$ pair. The single W contribution is overall less ``signal-like" to other $qq\ell\nu$ final states but assists in reducing the overall statisical uncertainty in a W mass measuement. The hadronic mode is more challenging due to the combinatoric assignment of the four hadronic jets into two W's along with color-reconnection which may cause ``cross-talk" between jets. The leptonic channel is also difficult because of the presence of two neutrinos and smaller branching fractions.

  The most difficult reconstruction of a final state from a semileptonic W decay is the case involving the tau lepton. The tau final states are equally as important as the light leptons because the W couples to the three lepton flavors democratically. The tau can mimic the signature of hadrons or other leptons in a detector in addition to producing additional missing energy via neutrinos.  The tau lepton mainly decays hadronically -- into a tau neutrino and virtual W-boson that produces a pair of quarks. The virtual W's daughter quarks will hadronize into a charged particle ($\pi^\pm$) or form more charged or neutral particles ($\pi^0$). The virtual W in the tau decay is allowed to couple with leptons, so, the tau final state can include either an electron or muon along with the corresponding flavor neutrinos. The decay rates for the tau are given in Table \ref{tab:taudecay}.
\begin{figure}
\centering
\feynmandiagram [horizontal=a to b] {
  i1 [particle=\(e^{-}\)] -- [fermion] a -- [fermion] i2 [particle=\(e^{+}\)],
  a -- [photon, edge label=\(\gamma / Z\)] b,
  f1 [particle=\(W^{+}\)] -- [photon] b -- [photon] f2 [particle=\(W^{-}\)],
};
    \feynmandiagram[vertical'=a to b ]{
        i1 [particle=\(e^{-}\)]
            -- [fermion] a 
            -- [boson] f1 [particle=\(W^{-}\)],
        a -- [fermion, edge label'=\(\nu_e\)] b ,
        i2 [particle=\(e^{+}\)]
            -- [anti fermion] b
            -- [boson] f2 [particle=\(W^{+}\)]
    };
\caption{\label{fig:wwdiag} Main WW production modes in the $s$ and $t$ channels. }
\end{figure}
\begin{table}
\centering
 \begin{tabular}{|c|l|c|} 
 
 \hline
       & Decay Mode & Branching Ratio  \\ \hline \hline
    Hadronic Modes  & $\pi^- \nu_\tau$  & $10.82\%$  \\
      	($64.79\%$) & $\pi^- \pi^0 \nu_\tau$ & $25.49\%$ \\
     				& $\pi^- \pi^0 \pi^0 \nu_\tau$  & $9.26\%$  \\
     				& $\pi^- \pi^0 \pi^0 \pi^0 \nu_\tau$  & $1.04\%$   \\
      				& $\pi^- \pi^+ \pi^- \nu_\tau$  & $8.99\%$      \\ 
      				& $\pi^- \pi^+ \pi^- \pi^0 \nu_\tau$  & $2.74\%$  \\ \hline
    			    
    Leptonic Modes  & $e^- \nu_e \nu_\tau$ & $17.82\%$   \\
    	($35.21\%$)	& $\mu^- \nu_\mu \nu_\tau $  & $17.39\%$      \\ \midrule \hline
      				
     				
\end{tabular}
        \caption{\label{tab:taudecay}Most common decay modes for the $\tau^-$ lepton \cite{pdg}}
\end{table}
       
  
   






  
  Measurements in the WW channel through  $e^+ \, e^-$ annihilation presents two important effects: (1) contributions from foreign particles which are reconstructed alongside the primary interaction and (2) the rate at which the W-pairs are produced, which is sensitive to beam helicities. The first effect has a contribution from two components, beamstrahlung and overlay. In beamstrahung photons are produced from the interactions between the fields of the beams. The radiated photons generally go undetected by escaping down the beam-pipe causing the effective center of mass energy to be reduced at the interaction point. The photons interact with other photons at a rate of 1.1 events per beam crossing \cite{ILDIDR} creating what is known as the $\gamma \gamma$ overlay. The $\gamma \gamma \rightarrow \text{hadrons}$ may scatter into the detector and add a source of confusion wherein the foreign particles ``overlay" on top of the true particles of an event. For the second effect, the W-pair production rate, there are four possible combinations of electron and positron helicities where each initial particle is either left or right handed. More explicitly, a collision can consist of  $e^-_L e^+_R$(LR) with left-handed electron and right-handed positron , $e^-_R e^+_L$(RL)  with right handed electron and left handed positron, or mirroring helicities RR and LL.  The beams in the collider are mixed with multiple helicities which is represented by overall partial longitudinal beam polarizations $P_{e^-}$ and $ P_{e^+}$.  Two W-bosons can only be produced in LR and RL configurations, whereas in the mirrored configurations, the recombination into a particle of spin 1 is  not possible.  The W has pure coupling only to left handed electrons or right handed positrons in the $t$-channel, so, the number of WW events produced are sensitive to the beam polarization\cite{thomson}.  If the number of events produced is sensitive to polarization then the overall cross-section for WW production is sensitive to beam polarization. 











