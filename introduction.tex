The W-boson pair decaying semileptonically is a rich physics channel with a wide variety of facets to study. The W mass measurement is well motivated because it is an essential fundamental parameter of the standard model. The WW channel is a unique way to pin down the W mass because of the easy identification of the charged lepton which automatically tags the hadronic W as the remaining particles in the system. The mass measurement quality is then only bounded by the performance of the detector and statistics. WW production is also sensitive to the polarization of electron positron collider beams due to neutrino coupling. Thus,  measuring the production cross section provides an opportunity to implicitly measure the beam polarization at the interaction point. Another important aspect of WW, is the charged triple gauge couplings(TGCs). Deviation of these couplings from the Standard Model is a distinct signature of new physics. This study assesses the challenges associated with reconstruction and analysis of semileptonic W pairs at center of mass energy 500 GeV, which is an important tool in understanding the problems that lie in analysis within the next frontier of particle physics. The work presented contains four major steps. The first step is the identification of the lepton, which is done with a universal treatment of leptons, that is, without distinguishing between lepton flavors.  Secondly, the effects of $\gamma\gamma$ overlay on the hadronic mass are explored and a technique to reduce this effect is showcased. Next is the event selection where W pairs are selected against the full standard model background. Lastly, estimates for the statistical error of the W mass and cross section are extracted.