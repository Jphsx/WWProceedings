\subsection{Analysis Overview}
\label{subsec:ana_overview}

The analysis workflow for semileptonic WW has three distinct stages, the lepton identification and selection, overlay rejection in the hadronic system, and event selection against full standard model backgrounds. The analysis is performed with the  running scenario that creates the most dominant WW production mode $(-0.8,+0.3)$. 


\subsection{Lepton Identification}
\label{subsec:Lepton_ID}
The approach towards the identification of leptons relies on treating leptons universally. The easiest lepton to identify is the muon, which produces a single track along with hits in the muon detector. The electron also produces a track in the TPC but is often accompanied by photons via bremsstrahlung. The tau is the most difficult lepton to identify due to its frequent decay into multiple charged and neutral particles. To accommodate all types of lepton signature, a cone based approach is used to either capture single tracks or collimated jets with low track multiplicity. The lepton finding cone is based on the TauFinder package \cite{taufinder} developed initially for the Compact Linear Collider(CLIC) studies. TauFinder consists of two major structures, a search cone containing the particles that belong to the lepton candidate and an isolation cone whose purpose is to reject a lepton candidate if the search cone is not well isolated from other particles. The acceptance criteria for a search cone consists of these parameters:
\begin{itemize}
\item Search cone angle $\alpha$ - The opening (half) angle of the search cone for the lepton jet [rad]
\item Isolation cone angle $\beta$ - The outer isolation cone angle w.r.t to the search cone [rad]
\item Isolation energy - The total energy allowed within the isolation cone region [GeV]
\item Invariant Mass - The upper limit on the lepton candidate mass [GeV]
\item Track multiplicity - The allowed number of tracks in a lepton candidate
\item Minimum $P_T$ seed - the minimum transverse momentum of a track that seeds a lepton candidate [GeV] 
\end{itemize}
An example of the cone and parameters are shown in Figure 2. Additional requirements are imposed on all of the reconstructed Particle Flow Objects(PFOs) in the event in order to suppress overlay particles being included in the lepton jet.
\begin{itemize}
\item $P_T > 0.2$ GeV
\item $|\cos\theta| < 0.99$
\end{itemize}
The formation of a lepton candidate follows three steps (1) candidate construction, (2) candidate merging, and (3) isolation testing.
The first step starts with seed tracks that are sorted by energy in descending order. Any track or neutral particle that falls within the search cone of the lepton candidate is added to the lepton candidate. For each newly added particle, the energy and momentum of the lepton candidate is updated. Each candidate has a unique set of particles. Lepton candidates are continually formed until the seed tracks are exhausted. When there are no more candidates to be created, the candidates are subjected to part of the acceptance criteria: the lepton jet mass is required to be below the upper mass limit (2 GeV) and the number of charged tracks within the lepton candidate is non-zero and less than or equal to 4. If a lepton jet violates any acceptance condition it is deleted. The next step in the process is merging. If two lepton candidates form an opening angle of less than $2\alpha$, the candidates are merged. If the mass or track multiplicity conditions are violated, both lepton candidates are deleted.  All  candidates that survive merging are subjected to the isolation testing. For each candidate, the sum of energy of all the particles that fall inside the isolation cone is computed. If the total energy inside the isolation cone is greater than the maximum allowed energy inside the isolation cone the lepton candidate is deleted.\\

	The universal lepton treatment is not conducive to a one size fits all approach to lepton ID due to the abundance of different lepton signatures. To accommodate variations between lepton signatures, the acceptance criteria for leptons is optimized according to lepton flavor and $\tau$ decay mode. The categories created are: Prompt $\ell$, $\tau \rightarrow \ell\bar{\nu_{\ell}} \nu_{\tau}$, $\tau \rightarrow $hadrons. The leptonic categories, for both prompt and $\tau$ decays, are further subdivided between light flavors whereas the hadronic category is separated by either 1-prong or 3-prong decays.\\


\begin{figure}[hp]
\centering
\captionsetup{justification=raggedright,margin=3cm}
\includegraphics[width=0.4\textwidth]{cone.pdf}
%\captionof{figure}{Illustration of possible lepton candidate cone with search cone angle $\alpha$ and isolation cone angle $\beta$. The search cone is shown in green and the isolation cone is the surrounding blue cone.}
\caption{Illustration of possible lepton candidate cone with search cone angle $\alpha$ and isolation cone angle $\beta$. The search cone is shown in green and the isolation cone is the surrounding blue cone.}
\end{figure}


 The optimal selection criteria for each category is the set of parameters that maximally identify lepton candidates that originate from true leptons and minimize the fake lepton candidates that originate from hadronic jets. To find this set of parameters, a scan over a 3D space is performed using the search cone-$\alpha$, isolation cone-$\beta$, and isolation energy-$E_{\text{iso}}$. The invariant mass is held at a fixed 2 GeV for simplicity.\\
Two parameters are defined to find the optimal working point in the lepton finding space. The first is related to correctly identifying jets originating from true leptons.
The true lepton reconstruction efficiency is maximized with the signal sample $WW\rightarrow q\bar{q}\ell\nu$ and denoted as  $\epsilon_T$. The denominator, $N_{\text{Stotal}}$, represents the total, category specific, number of events which contain three generator visible fermions. The truth $q\bar{q}\ell$ fermions are required to fall within the acceptance range $|\cos\theta| < 0.99$. $N_{\text{match}}$ is the number of signal sample events in which a lepton candidate is reconstructed and can be matched to the true lepton, such that the opening angle between the reconstructed lepton and the true lepton is less than 0.1 radians. The distribution of opening angles is show in Figure \ref{fig:taupsi}. In the case that a reconstructed lepton is being matched to a true tau, the matching angle is formed between the reconstructed lepton and the vector sum of the visible generator components of the tau decay. The visible components of the tau decay consist of the direct decay products whereas photons from final state radiation of the tau prior to decay are excluded.

\begin{equation}
\label{eq:et}
\epsilon_T = N_{\text{match}}/N_{\text{Stotal}}
\end{equation}

 The second optimization parameter is denoted as $P_F$, the probability of a fake lepton jet arising from a single hadronic jet. The fake lepton probability $P_F$ is minimized using the background sample $WW\rightarrow q\bar{q}q\bar{q}$ and is a function of the fake lepton reconstruction efficiency $\epsilon_F$.
\begin{equation}
\label{eq:ef}
\epsilon_F = N_{\text{fake}}/N_{\text{Btotal}}
\end{equation}
The $\epsilon_F$ denominator, $N_{\text{Btotal}}$ represents the total number of background events and is also subjected to the same acceptance range $|\cos\theta| < 0.99$ for all four fermions. The numerator, $N_{\text{fake}}$, is the total number of events  that contain at least one reconstructed fake lepton. The fake efficiency can be interpreted as the binomial probability of $r$-successes(lepton reconstructions) in 4 trials(hadronic jets). The probability of a single success in a single trial, $P_F$, can be directly derived from the binomial p.d.f using the fake efficiency $\epsilon_F$ per hadronic jet. 
\begin{equation}
\label{eq:pf}
P_F = 1-(1-\epsilon_F)^{\frac{1}{4}} 
\end{equation}
The optimal parameters $\alpha$, $\beta$, $E_{\text{iso}}$ for each lepton category are extracted from max$[\epsilon_T(1-P_F)]$. The results for each category are shown in Table \ref{tab:taufinderopt}. 

\begin{table}

\begin{tabular}{|p{0.14\textwidth}|p{0.14\textwidth}p{0.14\textwidth}p{0.14\textwidth}p{0.1\textwidth}p{0.1\textwidth}p{0.1\textwidth}|}

\hline 
Channel & $n\,\text{Lep}\geq 1$ $\,\, (\%)$  & $\epsilon_T \, \, (\%)$ & $1-P_{F} \, \, (\%)$ & $\alpha$ [rad] & $\beta$ [rad] & $E_{\text{iso}}$ [GeV] \\ 
\hline 
Prompt $\mu$ & $95.5 \pm 0.3$  & $94.9 \pm 0.3$ & $97.4 \pm 0.1$& 0.03 & 0.15 & 3.0 \\ 

Prompt $e$ & $92.0 \pm 0.3$  & $90.4 \pm 0.3$ & $96.1 \pm 0.1$ & 0.04 & 0.15 & 4.0 \\ 

Inclusive $\tau$ & $80.0 \pm 0.5$  &  $77.0 \pm 0.6$ & $94.3 \pm 0.1$& 0.07 & 0.15 & 4.5 \\ 


 \hline
$\tau \rightarrow \nu \nu \mu$ & $81.5 \pm 1.2$  & $80.1 \pm 1.3$ & $97.4 \pm 0.1$ & 0.03 & 0.15 & 3.0 \\ 
 
$\tau \rightarrow \nu \nu e$  &  $80.0 \pm 1.2$&   $78.1 \pm 1.3$ & $96.3 \pm 0.1$  & 0.05 & 0.15 & 3.5 \\ 
 
$\tau$ Had-1p & $74.4 \pm 0.9$  & $70.7 \pm 0.9$ & $93.0 \pm 0.2$ & 0.07 & 0.15 & 4.5 \\ 
 
$\tau$ Had-3p &  $75.6 \pm 1.5$  & $71.0 \pm 1.6$ & $93.0 \pm 0.2$ & 0.07 & 0.15 & 5.5  \\
\hline
\end{tabular} 
\caption{Optimization results using $100 \%$ LR $q\bar{q}\ell \nu$ and $q\bar{q} q\bar{q}$ samples. The $n \, \, \text{Lep} \geq 1$ column pertains to signal samples where at least one lepton candidate was found and is not subjected to the truth matching criterion of 0.1 radians. Results shown are the configurations that maximize $\epsilon_T(1-P_F)$. The Prompt $\mu$ and Inclusive $\tau$ cones are chosen for a tight and loose lepton selection respectively. }
\label{tab:taufinderopt}
\end{table}
  


\begin{figure}
\centering
    \begin{minipage}{0.48\textwidth}
        \centering

%\includegraphics[width=0.9\textwidth]{matchingangle.pdf}
\begin{tikzpicture}
    \node(a){\includegraphics[width=0.99\textwidth]{matchingangle.pdf}};
    \node at (a.north west)
    [
    anchor=center,
    xshift= 2.5cm,
    yshift= -1cm
    ]
    {
        \includegraphics[width=0.13\textwidth]{ildlogo.png}
    };
    \end{tikzpicture}
\caption{Distribution of opening angles between the closest reconstructed lepton candidate and the true muon from $WW \rightarrow q \bar{q} \mu \nu_\mu$. $99.4\%$ of events with a muon candidate are matched to truth.} 
\label{fig:taupsi}
\end{minipage}\hfill
    \begin{minipage}{0.48\textwidth}
        \centering

%\includegraphics[width=0.9\textwidth]{candEnergy.pdf}
\begin{tikzpicture}
    \node(a){\includegraphics[width=0.99\textwidth]{candEnergy.pdf}};
    \node at (a.north west)
    [
    anchor=center,
    xshift= 2.5 cm,
    yshift= -1cm
    ]
    {
        \includegraphics[width=0.13\textwidth]{ildlogo.png}
    };
    \end{tikzpicture}
\caption{Energy distribution of lepton candidates matched to truth from $WW \rightarrow q \bar{q} \mu \nu_\mu $ and fake candidates from $ WW \rightarrow q\bar{q} q \bar{q}$ both normalized to unity.\\}
\label{fig:candE}
\end{minipage}
\end{figure}





Since only one lepton is expected from signal, a single lepton candidate is selected as the candidate for the event. If multiple lepton jets are reconstructed then the lepton candidate with the highest energy is selected as the single candidate for the event. Any additional lepton candidates are treated as part of the hadronic system. The energy distribution of true and fake leptons is shown in Figure \ref{fig:candE}.


\subsection{$\gamma\gamma$ Overlay Mitigation}
\label{subsec:Pileup_mitigation}
\begin{figure}

\includegraphics[width=0.48\textwidth]{hxLR.pdf}
\includegraphics[width=0.48\textwidth]{hxRL.pdf}
\caption{The fractional energy partitioning of the true fermions with $\ell = \mu,\tau$, $100\%$ polarization for the initial state helicities LR (Left) and RL (Right) at center of mass energy 500 GeV. In the LR configuration the charged lepton and down-like quark take the majority of the beam energy. In RL configuration the energy partitioning is similar among the four fermions. }
\label{fig:Epartition}
\end{figure}

\begin{figure}


\includegraphics[width=0.48\textwidth]{hsLR.pdf}
\includegraphics[width=0.48\textwidth]{hsRL.pdf}
\caption{The Signed sine of the polar angle of the true fermions with $\ell = \mu,\tau$. $100\%$ polarization for the initial state helicites. The sign of $\sin\theta$ corresponds to the sign of $\cos\theta$. As $\sin\theta \rightarrow 0$ the fermion is forward in the detector but for $|\sin\theta| \rightarrow 1$ the fermions become maximally transverse to the beam. In the LR configuration (left) the charged lepton and down-like quark are scattered forward while the up-like quark and neutrino are ejected centrally into the detector. In the RL configuration (right) all of the fermions are more centrally produced. }
\label{fig:fangles}
\end{figure}

After a lepton candidate has been selected, the remaining particles in the system are expected to form the hadronically decaying W boson. However, hadronic mass is often in excess of the true hadonic mass. Variation between the true and measured mass naturally arises due to the mismeasurement of particles -- especially neutral hadrons, as well as particles lost beyond the acceptance of the detector. This effect is not substantial enough to account for the systematic excess observed in reconstruction. The nature of the excess in mass can be understood through the kinematics of the WW in the LR and RL configurations shown in Figures \ref{fig:Epartition} and \ref{fig:fangles}. The highest yielding configuration, LR, typically has two fermions that are forward in the detector which both typically have a large fraction of the beam energy. These fermions are susceptible to $\gamma\gamma$ scattering into the detector and mixing directly into the reconstructed jets. To combat effects of the $\gamma \gamma$ overlay, jet clustering algorithms via FastJet\cite{fastjet} are used.  The standard approach for overlay mitigation is to use the $k_T$ algorithm\cite{kt} and tune the R parameter such that the overlay particles are associated with beam jets while the desired particles are not. With successful $k_T$ clustering the beam jets can be thrown away without damaging the reconstruction of the desired event. However, this approach only works well in events that are centrally produced.  The overlay overlap in the forward topology with the $k_T$ algorithm  based clustering leads to rejecting desired particles and severe undermeasurment of the W mass. The solution to proper overlay mitigation is through the precise removal of foreign particles inside the reconstructed jets.  This can be achieved by using the standard JADE algorithm and cut-off parameter $y_{\text{cut}} > y_{ij}$ where $y_{ij} = d_{ij} / Q^2$ with $d_{ij} = 2E_i E_j(1-\cos\theta_{ij})$ \cite{fastjet}.  The mass of individually reconstructed jets can be controlled by tuning the $y_{\text{cut}}$ parameter. For large values, $y_{\text{cut}} =1\times10^{-3}$, a single massive jet is reconstructed. In the limit that $y_{\text{cut}}$ becomes infinitely small the number of jets reconstructed converges to the number of reconstructed particles.  The best $y_{\text{cut}}$ value is the value that forms mini-jets that safely couple together hard and soft emissions from  the original quark jet while segregating overlay into its own mini-jets. The mini-jets are then subjected to kinematic cuts that maximize the overlay rejection and minimize the difference  between the true and measured hadronic W mass. 	The best combination of $y_{\text{cut}}$ and mini-jet kinematic cuts are found by examining the $100\%$ polarized LR signal dataset for $qq \mu \nu$.  Two statisical estimators are used to maximize the overlay rejection, both of which come from the distribution of $M_{qq}^{\text{meas}} - M_{qq}^{\text{true}}$. This binned mass difference distribution is created from the subset of mini-jets that arise from clustering with a given $y_{\text{cut}}$ and also pass some jet requirements $P_{T} > x$ and $|\cos\theta| < y$. The estimators, from the distribution, are the Full Width Half Maximum(FWHM) and the number of entries in the Mode.  
Using estimators calculated from a binned histogram creates unwanted sensitivity to bin size. To reduce sensitivity to binning, firstly, the mode is defined as the bin with the most entries. The ``mode entries" is defined as the number of entries in the mode bin plus the number of entries in the nearest neighbors of the mode bin. For the FWHM, the mass distribution is assumed to be monotonically decreasing around the half maxmimum. To create a more sensitive continuous distribution of the FWHM,  the FWHM is weighted towards the bin center of the two bins around (above/below) the half maximum. The results of the optimization are shown in Figure \ref{fig:supmass} and various $y_{\text{cut}}$'s are shown in comparison to the optimal configuration in Figure \ref{fig:supdiff}. The optimal result uses $y_{\text{cut}} = 5\times 10^{-5}$, mini jet $P_T > 2$ GeV, and has no $|\cos \theta|$ requirement. 

\begin{figure}
    \centering
 %   \begin{minipage}{0.49\textwidth}
        \centering
        
        %\includegraphics[width=0.99\textwidth]{SupDiff.pdf} % first figure itself
        \begin{tikzpicture}
    %\node(a){\includegraphics[width=0.99\textwidth]{SupDiff.pdf}};
   \node(a){\includegraphics[scale=.45]{SupDiff.pdf}};
    \node at (a.north west)
    [
    anchor=center,
    xshift= 3.3cm,
    yshift= -1.3cm
    ]
    {
        \includegraphics[width=0.06\textwidth]{ildlogo.png}
    };
    \end{tikzpicture}
        \caption{Comparison of generator and reconstruction mass differences with different $y_{\text{cut}}$ values and the same  flat mini-jet cut of $P_T > 2$. The generator mass is computed from the vector sum of true di-quark pair from the hadronic W boson. The results of the largest $y_{\text{cut}}$ are massive (high $P_T$) jets that are not separated from overlay and insensitive to small kinematic cuts. $y_{\text{cut}} = 5\times 10^{-5}$ has the best balance between jet clustering and mini-jet cuts. $y_{\text{cut}} = 5\times 10^{-6}$ yields a highly fragmented system where the mini-jets are not distinguishable from overlay. The most fragmented system is sensitive to the $P_T$ requirement resulting in the small peak around -80 GeV where the hadronic W is completely thrown out.  }
        \label{fig:supdiff}
       \end{figure}
       
   % \end{minipage}\hfill
   % \begin{minipage}{0.49\textwidth}
   \begin{figure}
        \centering
       
        %\includegraphics[width=0.99\textwidth]{SupMass.pdf} % second figure itself
        \begin{tikzpicture}
    %\node(a){\includegraphics[width=0.99\textwidth]{SupMass.pdf}};
    \node(a){\includegraphics[scale=.45]{SupMass.pdf}};
    \node at (a.north west)
    [
    anchor=center,
    xshift= 3.3cm,
    yshift= -1.3cm
    ]
    {
        \includegraphics[width=0.06\textwidth]{ildlogo.png}
    };
    \end{tikzpicture}
        \caption{The increase of quality of the hadronic mass is shown between the red curve which is the raw hadronic system after lepton identification versus the black curve which is subjected to the overlay mitigation with $y_{\text{cut}} = 5\times10^{-5}$ and mini jet $P_T > 2$ GeV. On average the excess in mass is reduced by $\approx 5$ GeV. }
        \label{fig:supmass}
 %   \end{minipage}
\end{figure}


\subsection{Event Selection}
\label{subsec:EventSelection}
The W-pair selection has been optimized for a Monte Carlo sample of 1600 $\text{fb}^{-1}$ with the $(-0.8,+0.3)$ beam scenario and builds on those described in \cite{ivan}. However, there is a significant difference between the present and preceding analysis selections, such that  the previous selection only addresses prompt muons as signal and treats muonic tau decays as background. The selection includes the full 2, 4, 6 fermion and Higgs SM background, and is performed with two mutually exclusive subsets, a tight and loose selection. The tight selection uses the prompt muon cone to identify signal events that contain both prompt and non-prompt muons and electrons. The tight selection is inefficient in collecting hadronic taus, so, the loose selection, using the inclusive tau cone, is designed to recover the efficiency of hadronic taus and other problematic events. The tight and loose cone parameters are given from the optimization in Table \ref{tab:taufinderopt}. The selection criteria are applied after the overlay mitigation and are as follows:
\begin{itemize}
\item N Leptons $\geq 1$
\item Track Multiplicity $> 10$  
\item Visible $P_T > 5$ GeV  
\item $E_{\text{vis}} < 500$ GeV 
\item $E_{\text{com}} > 100$ GeV
\item $40<M_{qq}<120$ GeV
\item  $-q\cos\theta_W > -0.95$
\item  $(m^{\text{vis}}_{\text{recoil}})^2 < 135,000 \, \, \text{GeV}^2$
\end{itemize}

The track multiplicity and $E_{\text{com}}$ target 2 fermion backgrounds. $E_{\text{com}}$ is the rest-frame energy that consists of the visible and inferred missing energy. The missing energy is treated as a single neutrino with zero mass such that $E_{\text{com}} = E_{\text{vis}} + |P_{\text{miss}}| \, \,  \, \text{and} \, \, P^\mu_{\text{miss}} = (|P_{\text{miss}}| , -\sum{\vec{p}_{\text{vis}}})$. The visible $P_T$ is the magnitude of the vector of all measured transverse momenta and $E_{\text{vis}}$ is the sum of all reconstructed visible energy in an event. The $P_T$ and $E_{\text{vis}}$ cuts target processes that do not have a genuine missing energy from a neutrino. The hadronic W-mass $M_{qq}$ requirement forces the hadronic system to be ``W-like" and the recoil mass, $m_{\text{recoil}}^{\text{vis}}$, uniquely requires the visible system to be recoiling against an invisible system with little to no mass. The recoil mass is defined as $(m_{ \text{recoil}}^{\text{vis}})^2 = s + M^2_{\text{vis}} - 2\sqrt{s}E_{\text{vis}} \, \, \text{and} \, \, M^2_{\text{vis}} = ( P^{\mu}_{qq} +  P^{\mu}_{\ell})^2$. The W-scattering angle $-q\cos\theta_W$ is the angle of deflection of the system identified as $W^-$ with respect to the $e^-$ beam axis and is implemented to limit backward scattering.  The charge of the lepton is extracted from the leading momentum track from candidates with 1, 2, or 4 tracks, or in the case of 3 tracks, the charge is the sum of the three track charges. The hadronic system is then tagged with the opposite charge.  This procedure results in the correct charge assignment of the lepton before selection cuts for about $98.9\%$ of prompt muons, $94.8\%$ of prompt electrons, and $95.9\%$ of taus. Following the event selection the correct charge assignment increases to $99.9\%$ for prompt muons, $98.3\%$ for prompt electrons, and $98.8\%$ for taus. The distributions for the tight selection are shown in Figure \ref{fig:cutflow}. The details of the tight and loose selection for $(-0.8,+0.3)$ at 1600$\text{fb}^{-1}$ are summarized in Table \ref{tab:selection}. The selections in Table \ref{tab:selection} differ by the veto of the tight lepton in the loose selection, where the preference for any event is to always choose a tight lepton over loose. The primary selection also includes only ``WW-like" signal, this type of signal is such that both of the true fermion pairs invariant masses are each within $\pm10$ GeV of the nominal W mass. If an event contains a fermion pair that is outside the WW-like range, it is designated as an off-shell (O.S.) event and is placed in a different category. The selection details for events from the O.S. category are given in Table \ref{tab:os}. The selection criterion applied to every category are optimized to maximize the efficiency and purity of the total signal for the tight selection with WW-like events. The results are summarized with the two selection cones with and without O.S. events in Table \ref{tab:summary}. The final results for the selection with no restrictions on W mass and both cones, yields a signal efficiency of $60\%$ with only $7\%$ of background contaminating the selected events. The W-boson invariant mass distribution after selection is shown with mass resolution in Figure \ref{fig:money}.


\begin{table}

\caption{The tight and loose selection for 1600 $\text{fb}^{-1}$ $(-0.8,+0.3)$ WW-like events. The tight selection is most efficient with prompt leptons with $69\%$ and $65\%$  for muons and electrons respectively, but struggles to efficiently reconstruct taus. The loose selection recovers $10\%$ of the tau efficiency and is inefficient for prompt leptons because of the tight lepton veto. The tight lepton veto enforces the orthogonality of the selections and gives preference toward better lepton candidates. The signal categories Base Evts. only include events in which the true W mass of both fermion pairs are within 10 GeV of the nominal W mass.}
\label{tab:selection}
 \scriptsize
 Tight selection with muon cone \\
   \begin{tabular}{|p{0.11\textwidth}|p{0.06\textwidth}p{0.06\textwidth}p{0.06\textwidth}|p{0.08\textwidth}|p{0.06\textwidth}p{0.06\textwidth}p{0.06\textwidth}p{0.06\textwidth}|p{0.08\textwidth}|}
\hline 
 $(\times 10^5)$  & $qq\mu\nu$ & $qqe\nu$ & $qq\tau\nu$ & Tot. Sig. & 2f & 4f & 6f & Higgs & Tot. Bkg. \\ \hline 
Base Evts & {38.7 } &  {38.9 } &  {39.0} & {117} &  {422} &  {322} &  {2.14} &  {4.12} & 750 \\ 

Lep. + Cone & {33.1 } &  {32.0 } &  {22.8} & {87.8} &  {115} &  {118} &  {1.63} &  {1.15} & 236 \\ 
 
$E_{\text{vis}}$ & {32.8 } &  {31.1 } &  {22.7} & {86.7} &  {106} &  {115} &  {1.62} &  {1.11} & 224\\ 
 
N Tracks & {31.9 } &  {30.3 } &  {22.1} & {84.3} &  {25.4} &  {25.9} &  {1.49} &  {0.889} & 53.7\\ 
 
$-q\cos\theta$ & {31.8 } &  {30.1 } &  {21.8} & {83.7} &  {21.9} &  {22.6} &  {1.44} &  {0.852} & 46.8\\ 
 
$M_{qq}$ $>$40 & {29.4 } &  {28.0 } &  {20.3} & {77.7} &  {11.3} &  {13.3} &  {1.42} &  {0.756} & 26.8\\ 
 
$M_{qq}$ $<$120 & {27.2 } &  {25.7 } &  {18.3} & {71.3} &  {5.68} &  {2.68} &  {0.202} &  {0.297} & 8.86\\ 
 
$E_{\text{com}}$ & {27.2 } &  {25.7} &  {18.3} & {71.3} &  {5.58} &  {2.65} &  {0.202} &  {0.296} & 8.73\\ 

$P_T$ vis. & {26.9 } &  {25.5} &  {18.1} & {70.5} &  {3.21} &  {2.37} &  {0.201} &  {0.294} & 6.08\\ 

$(m^{\text{vis}}_{\text{recoil}})^2$ & {26.9 } &  {25.4 } &  {18.0} & {70.3} &  {2.93} &  {2.02} &  {0.194} &  {0.223} & 5.36 \\ 
\hline 

 $\epsilon \, \, (\%)$ & $69.4 $ & $65.4 $ & $46.2$ &  $60.3 $ & $0.69 $ & $0.626 $ & $9.05 $ & $5.41 $ & $0.72$ \\ 
 
 			& $\pm 0.2$ & $\pm 0.2$ & $\pm 0.3$ & $\pm 0.2$ & $\pm 0.01$ & $\pm 0.008 $& $\pm 0.02$ & $\pm0.05$  & $\pm 0.01$\\
\hline
\end{tabular}
\quad \quad \\
Loose selection with tau cone\\
\begin{tabular}{|p{0.11\textwidth}|p{0.06\textwidth}p{0.06\textwidth}p{0.06\textwidth}|p{0.08\textwidth}|p{0.06\textwidth}p{0.06\textwidth}p{0.06\textwidth}p{0.06\textwidth}|p{0.08\textwidth}|}
\hline 
 $(\times 10^5)$  & $qq\mu\nu$ & $qqe\nu$ & $qq\tau\nu$ & Tot. Sig. & 2f & 4f & 6f & Higgs & Tot. Bkg. \\ \hline 
Base Evts & {38.7 } &  {38.9 } &  {39.0} & {117} &  {422} &  {322} &  {2.14} &  {4.12} & 750\\ 
 
Lep.+ Cone & {33.6 } &  {33.0 } &  {28.2} & {94.8} &  {130} &  {136} &  {1.77} &  {1.38} & 269\\ 

Tight Veto & {0.772 } &  {1.28 } &  {5.70} & {7.76} &  {19.3} &  {21.5} &  {0.161} &  {0.312} & 41.3 \\ 
 
$E_{\text{vis}}$ & {0.764 } &  {1.26 } &  {5.70} & {7.72} &  {18.2} &  {19.4} &  {0.154} &  {0.302} & 38.1\\ 

N Tracks & {0.737 } &  {1.21 } &  {5.54} & {7.49} &  {15.0} &  {16.4} &  {0.151} &  {0.271} & 31.8\\ 
 
$-q\cos\theta$ & {0.630 } &  {1.12 } &  {5.32} & {7.07} &  {11.1} &  {14.1} &  {0.145} &  {0.256} &25.6\\ 
 
$M_{qq}$ $>$ 40 & {0.492 } &  {0.972 } &  {4.86} & {6.33} &  {5.98} &  {13.0} &  {0.144} &  {0.233}& 19.4 \\ 

$M_{qq}$ $<$ 120 & {0.404 } &  {0.781 } &  {4.16} & {5.35} &  {2.58} &  {1.11} &  {0.0111} &  {0.124} & 3.83\\ 
 
$E_{\text{com}}$ & {0.404 } &  {0.781 } &  {4.16} & {5.34} &  {2.50} &  {1.10} &  {0.0111} &  {0.124} &3.74\\ 

$P_T$ vis. & {0.400 } &  {0.774 } &  {4.12} & {5.29} &  {1.17} &  {1.01} &  {0.0111} &  {0.123} &2.31\\ 
 
$(m^{\text{vis}}_{\text{recoil}})^2$ & {0.394 } &  {0.770 } &  {4.07} & {5.24} &  {1.02} &  {0.759} &  {0.0102} &  {0.0973}& 1.89 \\ 
\hline 
 $\epsilon \, \, (\%)$ & $1.02 $ & $1.98 $ & $10.5 $ &  $4.49 $ & $0.241 $ & $0.236 $ & $0.474 $ & $2.36 $ & $0.251$\\ 

  	     & $\pm 0.05$ & $\pm 0.07$ & $\pm 0.2$ & $\pm 0.07$ & $\pm 0.003$ & $\pm 0.004$ & $\pm 0.007$ & $\pm 0.02$ & $\pm 0.003$ \\

 \hline
 \end{tabular}
 

\end{table}


\begin{table}
\centering
\caption{The tight and loose selection for 1600 $\text{fb}^{-1} (-0.8,+0.3)$ non WW-like or off-shell (O.S.) events.  The Base
Evts. only include events in which the true W mass of both fermion pairs has a minimum absolute difference
with the nominal W mass of 10 GeV. The selection is not optimized for selecting these types of events so the overall efficiency is less, compared to the WW-like events.}
\begin{minipage}{0.49\linewidth}
\scriptsize
 Tight selection with muon cone \\
\begin{tabular}{|p{0.24\textwidth}|p{0.14\textwidth}p{0.14\textwidth}p{0.14\textwidth}|p{0.14\textwidth}|}
\hline
  $(\times 10^5)$& $qq\mu\nu$ & $qqe\nu$ & $qq\tau\nu$ & Tot. Sig. \\ 
\hline
Base Evts &{5.78 } & {38.8 } & {5.70} & 50.3\\ 
 
Lep. + Cone &{5.11 } & {22.7 } & {3.42} & 31.2\\ 

$E_{\text{vis}}$ &{5.08 } & {22.5 } & {3.41} & 31.0\\ 

N Tracks &{4.95 } & {21.9 } & {3.35}& 30.2\\ 
 
$-q\cos\theta$ &{4.94 } & {21.0 } & {3.31}& 29.3\\ 
 
$M_{qq}>40$ &{4.67} & {20.1 } & {3.14} & 27.9\\ 
 
$M_{qq}<120$ &{3.44 } & {18.1 } & {2.39} & 23.9\\ 

$E_{\text{com}}$ &{3.44 } & {18.1 } & {2.39} & 23.9\\ 
 
$P_T$ vis. &{3.40 } & {18.0 } & {2.36} & 23.8\\ 
 
$(m^{\text{vis}}_{\text{recoil}})^2$ & {3.40 } & {17.6 } & {2.34}& 23.3\\ 
\hline 
 $\epsilon \, \, (\%)$ & $58.8$ & $45.4$ & $41.1$ & $46.3$ \\ 
 						& $\pm 0.6$ & $\pm 0.1$ & $\pm 0.6$ & $\pm 0.1$\\
 \hline
\end{tabular}
%\end{table}
\end{minipage}


%\begin{table}

\begin{minipage}{0.49\linewidth}
\scriptsize
\quad \quad \\
Loose selection with tau cone\\
 \begin{tabular}{|p{0.24\textwidth}|p{0.14\textwidth}p{0.14\textwidth}p{0.13\textwidth}|p{0.14\textwidth}|}
\hline 
 $(\times 10^5) $ & $qq\mu\nu$  & $qqe\nu$ & $qq\tau\nu$ & Tot. Sig.\\ \hline 
Base Evts &{5.78 } & {38.8 } & {5.70} & 50.3\\ 
 
Lep.+ Cone &{5.15 } & {24.7 } & {4.26} & 34.1\\ 

Tight Veto &{0.082 } & {2.61 } & {0.88} & 3.57\\ 

$E_{\text{vis}}$ &{0.079 } & {2.61 } & {0.88} & 3.57\\ 

N Tracks &{0.076 } & {2.48 } & {0.86} & 3.42\\ 

$-q\cos\theta$ &{0.070 } & {2.31 } & {0.82} & 3.20\\ 
 
$M_{qq} > 40$ &{0.056 } & {1.38 } & {0.77} & 2.20\\ 

$M_{qq} < 120$ &{0.036 } & {1.20 } & {0.48} & 2.05\\ 
 
$E_{\text{com}}$ &{0.036 } & {1.18 } & {0.48} & 2.03\\ 
 
$P_T$ vis. &{0.036 } & {1.18 } & {0.48 }& 2.02\\ 

$(m^{\text{vis}}_{\text{recoil}})^2$ &{0.036 } & {0.92 } & {0.47} & 1.42\\ 
\hline 
 $\epsilon \, \, (\%)$ & $0.63 $ & $2.36$ & $8.3$ & $2.82$\\ 
  			& $\pm 0.1$ & $\pm 0.05$ & $\pm 0.4$ & $\pm 0.05$\\  
 \hline
 \end{tabular}

\end{minipage}
\label{tab:os}
 \end{table}


\begin{table}
\caption{Selection summary showing the number of background events that pass the event selection, efficiency, and purity for the tight selection or combined selections and with or without the off-shell contributions.  The O.S. contributions are less efficiently reconstructed so their inclusion reduces the overall signal efficiency but boosts the purity by increasing the base number of signal events that can be selected. Selection is performed with 1600 $\text{fb}^{-1}$ in $(-0.8, +0.3)$. }
\label{tab:summary}
 \begin{tabular}{ |p{0.12\textwidth}|p{0.12\textwidth}p{0.15\textwidth}|p{0.08\textwidth}|p{0.12\textwidth}p{0.17\textwidth}|p{0.08\textwidth}|} 
 \hline 
   &  \multicolumn{3}{|l|}{Tight Selection} &  \multicolumn{3}{|l|}{ Tight + Loose Sel.}  \\  \hline  
 & Sel. Total $(\times 10^5)$ & Efficiency $(\%)$ & Purity $(\%)$ & Sel. Total $(\times 10^5)$ & Efficiency $\,\,\,\,$ $(\%)$& Purity $(\%)$ \\ 
 \hline  
 Bkg. &  {5.36} & 0.72 $\pm$ 0.01 & --&  {7.25} & 0.967 $\pm$ 0.005 &--  \\ 
 Signal &  {70.3} & 60.3 $\pm$ 0.2 & 92.9  &  {75.5} & 64.6 $\pm$ 0.2 & 91.2 \\ 
 Sig.+O.S. &  {93.6} & 56.0 $\pm$ 0.1 & 94.6 &  {100.3} & 60.0 $\pm$ 0.1 & 93.3 \\ 
\hline 
\end{tabular} 
\end{table}

\begin{figure}[htpb]%cutflow
\caption{ Distributions of the selection criteria for the $qq\ell\nu$ tight signal region. All distributions, excluding N tight leptons, include only one preceding requirement such that there is at least 1 identified tight lepton. The signal selection only includes events that pass the tight muon cone requirement. Uses 1600 $\text{fb}^{-1}$ in $(-0.8,+0.3)$.}
\centering
    \begin{minipage}{0.49\textwidth}
        \centering
   		\begin{tikzpicture}
        \node(a){\includegraphics[width=0.99\textwidth]{nlep.pdf}}; % second figure itself
        \node at (a.north west)
        [
        anchor = center,
        xshift=6.4cm,
        yshift=-1cm
     	]
     	{
     		\includegraphics[width=0.08\textwidth]{ildlogo.png}
     	};
     	\end{tikzpicture}
    \end{minipage}\hfill
    \begin{minipage}{0.49\textwidth}
        \begin{tikzpicture}
        \node(a){\includegraphics[width=0.99\textwidth]{Evis.pdf}}; % second figure itself
        \node at (a.north west)
        [
        anchor = center,
        xshift=6.4cm,
        yshift=-1cm
     	]
     	{
     		\includegraphics[width=0.08\textwidth]{ildlogo.png}
     	};
     	\end{tikzpicture}
     \end{minipage}\\
     \begin{minipage}{0.49\textwidth}
        \centering
   		\begin{tikzpicture}
        \node(a){\includegraphics[width=0.99\textwidth]{trackmult.pdf}}; % second figure itself
        \node at (a.north west)
        [
        anchor = center,
        xshift=6.4cm,
        yshift=-1cm
     	]
     	{
     		\includegraphics[width=0.08\textwidth]{ildlogo.png}
     	};
     	\end{tikzpicture}
    \end{minipage}\hfill
    \begin{minipage}{0.49\textwidth}
        \begin{tikzpicture}
        \node(a){\includegraphics[width=0.99\textwidth]{mqcost.pdf}}; % second figure itself
        \node at (a.north west)
        [
        anchor = center,
        xshift=6.2cm,
        yshift=-1cm
     	]
     	{
     		\includegraphics[width=0.08\textwidth]{ildlogo.png}
     	};
     	\end{tikzpicture}
     \end{minipage}\\
     \begin{minipage}{0.49\textwidth}
        \centering
   		\begin{tikzpicture}
        \node(a){\includegraphics[width=0.99\textwidth]{mqq.pdf}}; % second figure itself
        \node at (a.north west)
        [
        anchor = center,
        xshift=6.4cm,
        yshift=-1cm
     	]
     	{
     		\includegraphics[width=0.08\textwidth]{ildlogo.png}
     	};
     	\end{tikzpicture}
    \end{minipage}\hfill
    \begin{minipage}{0.49\textwidth}
        \begin{tikzpicture}
        \node(a){\includegraphics[width=0.99\textwidth]{Ecom.pdf}}; % second figure itself
        \node at (a.north west)
        [
        anchor = center,
        xshift=6.4cm,
        yshift=-1cm
     	]
     	{
     		\includegraphics[width=0.08\textwidth]{ildlogo.png}
     	};
     	\end{tikzpicture}
     \end{minipage}\\
     \begin{minipage}{0.49\textwidth}
        \centering
   		\begin{tikzpicture}
        \node(a){\includegraphics[width=0.99\textwidth]{ptvis.pdf}}; % second figure itself
        \node at (a.north west)
        [
        anchor = center,
        xshift=6.4cm,
        yshift=-1cm
     	]
     	{
     		\includegraphics[width=0.08\textwidth]{ildlogo.png}
     	};
     	\end{tikzpicture}
    \end{minipage}\hfill
    \begin{minipage}{0.49\textwidth}
        \begin{tikzpicture}
        \node(a){\includegraphics[width=0.99\textwidth]{mvisrecoil.pdf}}; % second figure itself
        \node at (a.north west)
        [
        anchor = center,
        xshift=6.4cm,
        yshift=-1cm
     	]
     	{
     		\includegraphics[width=0.08\textwidth]{ildlogo.png}
     	};
     	\end{tikzpicture}
     \end{minipage}\\
     
\label{fig:cutflow}


\end{figure}


\begin{figure}

\centering
   \begin{minipage}{0.50\textwidth}
       \centering
      %  \includegraphics[width=0.99\textwidth]{moneymassdiff.pdf} % first figure itself
      \begin{tikzpicture}
  %  \node(a){\includegraphics[width=0.99\textwidth]{moneymassdiff.pdf}};
  %	\node(a){\includegraphics[scale=0.4]{moneymassdiff.pdf}};
  %\node(a){\includegraphics[scale=0.6]{mwhadCutsHist.pdf}};
  \node(a){\includegraphics[width=0.99\textwidth]{massAllCut.pdf}};
    \node at (a.north west)
    [
    anchor=center,
     xshift= 2.5cm,
    yshift= -1.25cm
    ]
    {
        \includegraphics[width=0.13\textwidth]{ildlogo.png}
    };
    \end{tikzpicture}
   
   
    \end{minipage}\hfill
    \begin{minipage}{0.50\textwidth}
        \centering
        \begin{tikzpicture}
       % \node(a){\includegraphics[width=0.99\textwidth]{mwhadCutsHist.pdf}}; % second figure itself
       %\node(a){\includegraphics[scale=0.42]{mwhadCutsHist.pdf}};
      % \node(a){\includegraphics[scale=0.6]{moneymassdiff.pdf}};
      \node(a){\includegraphics[width=0.99\textwidth]{gendiff_allcut_nocut.pdf}};
        \node at (a.north west)
        [
        anchor = center,
         xshift= 2.5cm,
    yshift= -1.25cm
     	]
     	{
     		\includegraphics[width=0.13\textwidth]{ildlogo.png}
     	};
    	\end{tikzpicture}
    \end{minipage}\\
     \caption{The left distribution shows the  mass of the hadronic W-boson after overlay removal and selection cuts against the remaining background events. The selection used is the tight selection. Off-shell events are not displayed. The right distribution shows the resolution of the hadronic W mass with respect to the true mass in Monte Carlo to illustrate the difference of the raw reconstruction before $\gamma \gamma$ removal and selection versus selected events proceeding $\gamma \gamma$ removal.  Uses 1600 $\text{fb}^{-1}$ in $(-0.8,+0.3)$. 
}
\label{fig:money}
\end{figure}

\subsection{Results}
\label{subsec:wmass}

A fit with the convolution of a Breit-Wigner with a Gaussian (Voigtian) of the hadronic W mass is performed on the tight signal sample with the combined lepton categories shown in Figure \ref{fig:badfit}. The resulting fit models the shape and the mean of the distribution well but deviates around 90 GeV and the edges of the fit window. The width of the fit is also in excess of the true width, which is about 2 GeV. This means that the Voigtian model is inadequate in describing the simulated data likely due to the inadequacy of a single Gaussian for modeling the resolution. However, because the shape is similar to simulated data, the fitted model is used to understand the achievable mass resolution given a perfect model. Statistics consistent with 1600 $\text{fb}^{-1}$ (9.36M Events) are produced according to the previously fitted model with a mean  $M_W = 79.7079$ GeV, width $\Gamma_W = 10.6972$ GeV and $\sigma_W = 0.0$ GeV and refitted to achieve the statistical error on the mean $\Delta M_W \text{(stat.)} = 2.4$ MeV with goodness-of-fit $\chi^2 / ndof = 67.8/77$. This fit neglects background contributions but includes the off shell contributions. The toy model refit is also included in Figure \ref{fig:badfit}.
The cross-section and errors are extracted from Table \ref{tab:selection} according to the formula:
\begin{equation}
\sigma = \frac{N_T - N_B}{L \epsilon}
\end{equation}
where $N_T$ is the observed number of events that passes the selection and $N_B$ is the expected number of background events that contaminate the signal selection. The resulting statistical error on the cross-section is dominated by  Poisson errors on $N_T$, with subleading contributions from L such that $\frac{\Delta L}{L} = 0.026 \%$ and polarization uncertainty contributing $0.0019\%$ overall to $\frac{\Delta L}{L}$ \cite{ilcluminosity}. With $\epsilon$, $L$, and $N_B$ known perfectly, the combination of the errors for efficiency and $N_T$, and neglecting $\Delta L$, the resulting cross-section obtained in the combined selection is $8001 \pm 17 \, \, \text{fb}$ with statistical error of $\frac{\Delta \sigma}{\sigma} = 0.036 \%$. 
\begin{figure}

\centering
    \begin{minipage}{0.49\textwidth}
        \centering
       % \includegraphics[width=0.99\textwidth]{WWSfit.pdf} % first figure itself
   		\begin{tikzpicture}
        \node(a){\includegraphics[width=0.99\textwidth]{WWSfit.pdf}}; % second figure itself
        \node at (a.north west)
        [
        anchor = center,
        xshift=6cm,
        yshift=-1.25cm
     	]
     	{
     		\includegraphics[width=0.13\textwidth]{ildlogo.png}
     	};
     	\end{tikzpicture}
    \end{minipage}\hfill
    \begin{minipage}{0.49\textwidth}
        \centering
        \includegraphics[width=0.99\textwidth]{Wtoyfit.pdf} % second figure itself
     
     \end{minipage}\\
     \caption{ The left distribution shows the $(-0.8,+0.3)$ W mass distribution for all signal after selection cuts. The fit results in the Voigtian parameters $M_W = 79.7079$ GeV, width $\Gamma_W = 10.6972$ GeV and $\sigma_W = 0.0$ GeV with Monte Carlo statistics scaled up to 1600 $\text{fb}^{-1}$. The right distribution shows the refit with 9.36M W's generated according to the fitted model.}
\label{fig:badfit}


\end{figure}


