



The software ecosystem for the ILC is contained under iLCSoft \cite{ilcsoft} version \url{v02-00-02} which is comprised of reconstruction tools that rely on the event data model LCIO\cite{lcio}. Full simulation samples that are generated are based on detector descriptions in DD4HEP \cite{dd4hep}. The physics hard events are centrally produced with the Whizard event generator \cite{ whizard} with hadronization performed in Pythia6 \cite{pythia6} then simulated in the detector with Geant4 \cite{geant4}. The Monte Carlo samples being used are for the Interim Design Report benchmarking effort \cite{ILDIDR} and are based on events created for the ILD Technical Design Report \cite{tdrdet}. The analysis relies on Monte Carlo events that are fully simulated using the ILD detector model \url{ ILD_l5_o1_v02 } and includes a complete standard model background for final states with 2, 4, and 6 fermions as well standard model Higgs production.  A center of mass energy of 500 GeV and longitudinally-polarized beams. The Monte Carlo events are generated for $100\%$ polarized beams. The $\gamma \gamma$ event contributions are overlaid on top of the hard event before reconstruction. Events are reconstructed with PandoraPFA \cite{particleflowA} into particle flow objects (PFOs). Each event is weighted in order to obtain a realistic case of partial polarization for a possible running with $-80\%$ for the electron beam and $+30\%$ for the positron beam, comprising an integrated luminosity of 1600 fb-1.  The analysis framework for this project can be found at \cite{wwrepo}. 


