%\documentclass[11pt,a4paper]{article}
%\usepackage[utf8]{inputenc}
%\usepackage{amsmath}
%\usepackage{amsfonts}
%%\usepackage{amssymb}
%\begin{document}

%\end{document}

 The study showcases the approaches towards lepton identification and pileup mitigation at center-of-mass energy $\sqrt{s} = 500$ GeV for semileptonic WW decays at ILC. The analysis is performed using fully simulated Standard Model Monte Carlo events with the ILD detector concept and emphasizes the measurement of the W mass. The mass measurement is performed through the identification of a lepton and treatment of the remaining system as the hadronic W-boson. Only the most favorable beam polarization scenario for WW production is considered. The resulting detector performance benchmark obtained with an integrated luminosity of $1600 \, \, \text{fb}^{-1}$ is the statistical error of the W mass $\Delta M_W\text{(stat.)} =  2.4 $ MeV and a relative statistical error on the WW cross-section $\Delta \sigma / \sigma \text{(stat.)} = 0.038\% $. 

