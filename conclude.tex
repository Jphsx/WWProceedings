The results show a promising start to potential electroweak precision measurements for the ILC with the statistical error on the mass of $\Delta M_W\text{(stat.)}= 2.4$ MeV. This measurement is competitive with the current PDG measurement of $\Delta M_W=12$ MeV.   Comparing with another ILC study of the W mass at $\sqrt{s} = 250$ GeV, the achievable statistical error is $\Delta M_W\text{(stat.)} = 1.6$ MeV for a Gaussian mass resolution $\sigma_W= 4$ GeV with $10^7$ hadronic W's\cite{graham}. The estimated uncertainties at ILC250 provide a limit on the resolution of the W-mass in ILC500 being $\Delta M_W =3.7$ MeV. The mass measurement at ILC500 is more challenging because the W pair overlapping with pile-up is not a major issue at lower of center of mass energies. The equivalent precision may be unobtainable at higher $\sqrt{s}$, but propagating the same systematic error on the W mass for this study yields $\Delta M_W = 4.6$ MeV. Both measurements, however, are dominated by the systematic uncertainties from the effective jet energy scale which is a challenging demand.  The statistical error on the cross-section also shows the utility of semileptonic WW as a method to precisely measure the beam polarization at the interaction point. This offers an important alternative for  correctly measuring processes that are sensitive to beam polarization and assists in quantifying beam depolarization from collisions.

The lepton identification and charge assignment performance is exceptional with a overall correct charge assignment with $98.8\%$ over all three channels. This has the biggest impact on the measurement of charged triple gauge couplings that rely on the identification of the $W^-$. The lepton identification itself, still has room for improvement in ways such as (1) an MVA type approach with TauFinder and (2) re-optimization of the parameters over a mixed polarization beam-scenario which involves both LR and RL events. Currently the optimzation was performed on a purely LR configuration. This helicity combination has a distinct advantage of well isolated leptons, whereas in RL a well separated lepton is not guaranteed. The optimization of TauFinder parameters also led to every isolation cone for each category to choose 150 mrad, which is indicative of not achieving a global minimum. However, allowing the isolation to grow wider could mean overtuning the lepton identification to the topologies specific to LR. 

The pileup mitigation is a mostly unexplored avenue of reconstruction in ILC500 as most processes are produced centrally or are not simulated with pile-up. The techniques developed to remove pile-up are optimized for W mass measurement, but, can be easily adapted to general usage in any type of process where the standard approaches for pileup removal are inadequate. 

The event selection can also be improved. Additional cuts were explored such as the hadronic recoil against the leptonic W, the leptonic W mass, or maximum track multiplicity. The leptonic W mass cut best motivates the categorization approach of WW-like and not WW-like types of events, but, this cut, and others mentioned do not improved the overall efficiency times purity of the analysis.  Specifically, the leptonic W mass can be improved and applied to event selection by using more sophisticated calculation for the neutrino momentum. This can be done taking into account potential ISR in the z-direction, and applying a kinematic fit with constraints on the energy, momentum, and equal W masses. These adjustments would significantly improve the measured leptonic W mass and enhance the performance of the event selection. 
 
Some additional detector benchmarks can immediately follow the results of this analysis, one would be the quality of separation between prompt muons and tau muons to evaluate the performance of the vertex detector. Another study that should be done is examination of the analysis efficiency as a function of the polar angle, which tests the performance of the forward calorimeters.  Overall, the semi leptonic analysis offers keen insights to analysis performed in an electron positron collider with $\sqrt{s} = 500$ GeV. The statistical errors on cross-section and mass are the first step but an important and tractable step in electroweak precision measurement at the ILC.  The semileptonic channel still offers significantly more important physics in terms of TGC and polarization measurements in addition to expanding and improving the analysis presented here. 